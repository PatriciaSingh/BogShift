% Options for packages loaded elsewhere
% Options for packages loaded elsewhere
\PassOptionsToPackage{unicode}{hyperref}
\PassOptionsToPackage{hyphens}{url}
\PassOptionsToPackage{dvipsnames,svgnames,x11names}{xcolor}
%
\documentclass[
  letterpaper,
  DIV=11,
  numbers=noendperiod]{scrartcl}
\usepackage{xcolor}
\usepackage{amsmath,amssymb}
\setcounter{secnumdepth}{-\maxdimen} % remove section numbering
\usepackage{iftex}
\ifPDFTeX
  \usepackage[T1]{fontenc}
  \usepackage[utf8]{inputenc}
  \usepackage{textcomp} % provide euro and other symbols
\else % if luatex or xetex
  \usepackage{unicode-math} % this also loads fontspec
  \defaultfontfeatures{Scale=MatchLowercase}
  \defaultfontfeatures[\rmfamily]{Ligatures=TeX,Scale=1}
\fi
\usepackage{lmodern}
\ifPDFTeX\else
  % xetex/luatex font selection
\fi
% Use upquote if available, for straight quotes in verbatim environments
\IfFileExists{upquote.sty}{\usepackage{upquote}}{}
\IfFileExists{microtype.sty}{% use microtype if available
  \usepackage[]{microtype}
  \UseMicrotypeSet[protrusion]{basicmath} % disable protrusion for tt fonts
}{}
\makeatletter
\@ifundefined{KOMAClassName}{% if non-KOMA class
  \IfFileExists{parskip.sty}{%
    \usepackage{parskip}
  }{% else
    \setlength{\parindent}{0pt}
    \setlength{\parskip}{6pt plus 2pt minus 1pt}}
}{% if KOMA class
  \KOMAoptions{parskip=half}}
\makeatother
% Make \paragraph and \subparagraph free-standing
\makeatletter
\ifx\paragraph\undefined\else
  \let\oldparagraph\paragraph
  \renewcommand{\paragraph}{
    \@ifstar
      \xxxParagraphStar
      \xxxParagraphNoStar
  }
  \newcommand{\xxxParagraphStar}[1]{\oldparagraph*{#1}\mbox{}}
  \newcommand{\xxxParagraphNoStar}[1]{\oldparagraph{#1}\mbox{}}
\fi
\ifx\subparagraph\undefined\else
  \let\oldsubparagraph\subparagraph
  \renewcommand{\subparagraph}{
    \@ifstar
      \xxxSubParagraphStar
      \xxxSubParagraphNoStar
  }
  \newcommand{\xxxSubParagraphStar}[1]{\oldsubparagraph*{#1}\mbox{}}
  \newcommand{\xxxSubParagraphNoStar}[1]{\oldsubparagraph{#1}\mbox{}}
\fi
\makeatother


\usepackage{longtable,booktabs,array}
\usepackage{calc} % for calculating minipage widths
% Correct order of tables after \paragraph or \subparagraph
\usepackage{etoolbox}
\makeatletter
\patchcmd\longtable{\par}{\if@noskipsec\mbox{}\fi\par}{}{}
\makeatother
% Allow footnotes in longtable head/foot
\IfFileExists{footnotehyper.sty}{\usepackage{footnotehyper}}{\usepackage{footnote}}
\makesavenoteenv{longtable}
\usepackage{graphicx}
\makeatletter
\newsavebox\pandoc@box
\newcommand*\pandocbounded[1]{% scales image to fit in text height/width
  \sbox\pandoc@box{#1}%
  \Gscale@div\@tempa{\textheight}{\dimexpr\ht\pandoc@box+\dp\pandoc@box\relax}%
  \Gscale@div\@tempb{\linewidth}{\wd\pandoc@box}%
  \ifdim\@tempb\p@<\@tempa\p@\let\@tempa\@tempb\fi% select the smaller of both
  \ifdim\@tempa\p@<\p@\scalebox{\@tempa}{\usebox\pandoc@box}%
  \else\usebox{\pandoc@box}%
  \fi%
}
% Set default figure placement to htbp
\def\fps@figure{htbp}
\makeatother





\setlength{\emergencystretch}{3em} % prevent overfull lines

\providecommand{\tightlist}{%
  \setlength{\itemsep}{0pt}\setlength{\parskip}{0pt}}



 


\KOMAoption{captions}{tableheading}
\makeatletter
\@ifpackageloaded{tcolorbox}{}{\usepackage[skins,breakable]{tcolorbox}}
\@ifpackageloaded{fontawesome5}{}{\usepackage{fontawesome5}}
\definecolor{quarto-callout-color}{HTML}{909090}
\definecolor{quarto-callout-note-color}{HTML}{0758E5}
\definecolor{quarto-callout-important-color}{HTML}{CC1914}
\definecolor{quarto-callout-warning-color}{HTML}{EB9113}
\definecolor{quarto-callout-tip-color}{HTML}{00A047}
\definecolor{quarto-callout-caution-color}{HTML}{FC5300}
\definecolor{quarto-callout-color-frame}{HTML}{acacac}
\definecolor{quarto-callout-note-color-frame}{HTML}{4582ec}
\definecolor{quarto-callout-important-color-frame}{HTML}{d9534f}
\definecolor{quarto-callout-warning-color-frame}{HTML}{f0ad4e}
\definecolor{quarto-callout-tip-color-frame}{HTML}{02b875}
\definecolor{quarto-callout-caution-color-frame}{HTML}{fd7e14}
\makeatother
\makeatletter
\@ifpackageloaded{caption}{}{\usepackage{caption}}
\AtBeginDocument{%
\ifdefined\contentsname
  \renewcommand*\contentsname{Table of contents}
\else
  \newcommand\contentsname{Table of contents}
\fi
\ifdefined\listfigurename
  \renewcommand*\listfigurename{List of Figures}
\else
  \newcommand\listfigurename{List of Figures}
\fi
\ifdefined\listtablename
  \renewcommand*\listtablename{List of Tables}
\else
  \newcommand\listtablename{List of Tables}
\fi
\ifdefined\figurename
  \renewcommand*\figurename{Figure}
\else
  \newcommand\figurename{Figure}
\fi
\ifdefined\tablename
  \renewcommand*\tablename{Table}
\else
  \newcommand\tablename{Table}
\fi
}
\@ifpackageloaded{float}{}{\usepackage{float}}
\floatstyle{ruled}
\@ifundefined{c@chapter}{\newfloat{codelisting}{h}{lop}}{\newfloat{codelisting}{h}{lop}[chapter]}
\floatname{codelisting}{Listing}
\newcommand*\listoflistings{\listof{codelisting}{List of Listings}}
\makeatother
\makeatletter
\makeatother
\makeatletter
\@ifpackageloaded{caption}{}{\usepackage{caption}}
\@ifpackageloaded{subcaption}{}{\usepackage{subcaption}}
\makeatother
\usepackage{bookmark}
\IfFileExists{xurl.sty}{\usepackage{xurl}}{} % add URL line breaks if available
\urlstyle{same}
\hypersetup{
  pdftitle={GBIF Occurrence Data: from search to map},
  pdfauthor={Patrícia Singh},
  colorlinks=true,
  linkcolor={blue},
  filecolor={Maroon},
  citecolor={Blue},
  urlcolor={Blue},
  pdfcreator={LaTeX via pandoc}}


\title{GBIF Occurrence Data: from search to map}
\author{Patrícia Singh}
\date{2025-09-24}
\begin{document}
\maketitle


\subsection{title: ``GBIF Occurrence Data: from search to map'' author:
``Patrícia Singh'' date: ``2025-09-24'' format: html: toc: true
toc-depth: 3 number-sections: true code-fold: show code-tools: true pdf:
default docx: default editor: visual jupyter: r execute: echo: true
warning: false message:
false}\label{title-gbif-occurrence-data-from-search-to-map-author-patruxedcia-singh-date-2025-09-24-format-html-toc-true-toc-depth-3-number-sections-true-code-fold-show-code-tools-true-pdf-default-docx-default-editor-visual-jupyter-r-execute-echo-true-warning-false-message-false}

\begin{quote}
This Quarto document (\texttt{.qmd}) is ready to knit to
\textbf{HTML/PDF/DOCX} and to publish on a website (Quarto Pub or GitHub
Pages). It cleans up and structures your commented script into a
reproducible workflow.
\end{quote}

\subsection{1. Packages \& setup}\label{packages-setup}

\begin{verbatim}
# Install (first time) and load required packages # install.packages(c("rgbif", "dplyr", "maps")) library(rgbif) library(dplyr) library(maps) 
\end{verbatim}

\subsubsection{What we'll do}\label{what-well-do}

\begin{enumerate}
\def\labelenumi{\arabic{enumi}.}
\item
\item
  Quick counts from GBIF (occurrences \& observations).
\item
\item
  Explore one species (\emph{Sphagnum fuscum}).
\item
\item
  Handle GBIF's 10k download limit: \texttt{occ\_search()} vs
  \texttt{occ\_download()}.
\item
\item
  Save the file and import the full dataset.
\item
\item
  Make a simple map.
\item
\end{enumerate}

\begin{tcolorbox}[enhanced jigsaw, bottomtitle=1mm, bottomrule=.15mm, opacitybacktitle=0.6, left=2mm, colframe=quarto-callout-tip-color-frame, colbacktitle=quarto-callout-tip-color!10!white, title=\textcolor{quarto-callout-tip-color}{\faLightbulb}\hspace{0.5em}{Tip: Occurrence vs Observation}, rightrule=.15mm, opacityback=0, colback=white, titlerule=0mm, toptitle=1mm, toprule=.15mm, arc=.35mm, leftrule=.75mm, breakable, coltitle=black]

\begin{itemize}
\item
\item
  \textbf{Occurrence} = any presence record (specimens, fossils,
  observations, eDNA\ldots).
\item
\item
  \textbf{Observation} = specifically field/machine observations
  (e.g.~iNaturalist, bird checklists, camera traps).\\

  :::
\item
\end{itemize}

\subsection{2. Quick counts from GBIF}\label{quick-counts-from-gbif}

\begin{verbatim}
# All GBIF occurrences (all bases of record) occ_count()  # Observations only occ_count(basisOfRecord = "OBSERVATION")  # Occurrences by country (ISO codes) occ_count(country = "CZ")  # Czech Republic occ_count(country = "SK")  # Slovakia  # Observations by country occ_count(country = "CZ", basisOfRecord = "OBSERVATION") occ_count(country = "SK", basisOfRecord = "OBSERVATION") 
\end{verbatim}

\subsection{\texorpdfstring{3. Example species: \emph{Sphagnum
fuscum}}{3. Example species: Sphagnum fuscum}}\label{example-species-sphagnum-fuscum}

\begin{verbatim}
# Check accepted name and key (synonyms) name_suggest(q = "Sphagnum fuscum", rank = "species")  # Quick look at available records with coordinates (preview 10) occ_search(   scientificName = "Sphagnum fuscum",   hasCoordinate   = TRUE,   limit           = 10 ) 
\end{verbatim}

\begin{quote}
\textbf{Heads‑up: \texttt{occ\_search()} limits}

\texttt{occ\_search()} will \textbf{not} return more than
\textbf{100,000} records in one call. If there are more, filter by time
(\texttt{year}), geography
(\texttt{decimalLatitude}/\texttt{decimalLongitude}), or tile the area
and loop. For truly large pulls, use \texttt{occ\_download()} below.\\
\end{quote}

\subsubsection{Filter to human observations
only}\label{filter-to-human-observations-only}

\begin{verbatim}
occ_search(   scientificName = "Sphagnum fuscum",   hasCoordinate  = TRUE,   basisOfRecord  = "HUMAN_OBSERVATION",   limit          = 10 ) 
\end{verbatim}

\subsubsection{\texorpdfstring{Pull up to 10k records with
\texttt{occ\_search()}}{Pull up to 10k records with occ\_search()}}\label{pull-up-to-10k-records-with-occ_search}

\begin{verbatim}
# Up to 10,000 records directly into R sph_occ <- occ_search(   scientificName = "Sphagnum fuscum",   hasCoordinate  = TRUE,   basisOfRecord  = "HUMAN_OBSERVATION",   limit          = 10000 )  # Extract the actual data frame sph_occ_df <- sph_occ$data 
\end{verbatim}

\subsection{\texorpdfstring{4. Large downloads with
\texttt{occ\_download()}
(recommended)}{4. Large downloads with occ\_download() (recommended)}}\label{large-downloads-with-occ_download-recommended}

For complete datasets (beyond 10k), request a \textbf{server-side} GBIF
download and then fetch it.

\subsubsection{\texorpdfstring{4.1. Create
\texttt{\textasciitilde{}/.Renviron} with GBIF credentials
(one-time)}{4.1. Create \textasciitilde/.Renviron with GBIF credentials (one-time)}}\label{create-.renviron-with-gbif-credentials-one-time}

\begin{verbatim}
# Where is your home directory (Windows example)? path.expand("~") # Create a text file named .Renviron in that folder with these 3 lines: # GBIF_USER=your_gbif_username # GBIF_PWD=your_gbif_password # GBIF_EMAIL=your_email@domain.com  # Restart R so Sys.getenv() can read them Sys.getenv("GBIF_USER") Sys.getenv("GBIF_EMAIL") 
\end{verbatim}

\begin{quote}
\textbf{Windows gotcha}

If \texttt{Sys.getenv()} returns empty strings, your file is probably
\texttt{*.Renviron.txt*}.

\begin{itemize}
\item
\item
  In \textbf{File Explorer} → \textbf{View → Show → File name
  extensions}, then rename to \textbf{\texttt{.Renviron}} (no
  \texttt{.txt}).\\

  :::
\item
\end{itemize}

\subsubsection{4.2. Submit the download
request}\label{submit-the-download-request}

\begin{verbatim}
# Get the GBIF taxonKey for Sphagnum fuscum key <- name_backbone(name = "Sphagnum fuscum")$usageKey  # Request a download (runs on GBIF servers) req <- occ_download(   pred("taxonKey", key),   pred("hasCoordinate", TRUE),   pred("basisOfRecord", "HUMAN_OBSERVATION"),   format = "SIMPLE_CSV"  # tidy columns ) 
\end{verbatim}

\subsubsection{4.3. Wait until it's ready, then fetch \&
import}\label{wait-until-its-ready-then-fetch-import}

\begin{verbatim}
# Block until GBIF marks it as finished occ_download_wait(req)  # Download the zip locally (set overwrite = TRUE if re-downloading) zip_path <- occ_download_get(req, overwrite = TRUE) zip_path  # character path to the downloaded .zip  # Import the CSVs inside the zip to a data.frame dat <- occ_download_import(zip_path)  # Quick checks nrow(dat) dplyr::count(dat, basisOfRecord) 
\end{verbatim}

Typical metadata echoed after \texttt{occ\_download\_wait()} include
\textbf{Status}, \textbf{DOI}, \textbf{Download key},
\textbf{Created/Modified} timestamps, and \textbf{Total records}. Keep
the DOI for citation.\\

\subsection{5. Quick mapping}\label{quick-mapping}

\begin{verbatim}
# Simple world map then points map("world") points(dat$decimalLongitude, dat$decimalLatitude, pch = 19, cex = .4)  # Europe extent example map('world', xlim = c(-25, 45), ylim = c(34, 72)) points(dat$decimalLongitude, dat$decimalLatitude, pch = 19, cex = .5) 
\end{verbatim}

To show the \textbf{entire world}, just use \texttt{map("world")}
without \texttt{xlim}/\texttt{ylim}. To focus on Europe, set
\texttt{xlim}/\texttt{ylim} as above.\\

\subsection{6. Reproducibility}\label{reproducibility}

\begin{verbatim}
sessionInfo() 
\end{verbatim}

\subsection{7. Publishing options}\label{publishing-options}

\begin{itemize}
\item
\item
  \textbf{Quarto Pub}: \texttt{quarto\ publish} directly from RStudio/VS
  Code.
\item
\item
  \textbf{GitHub Pages}: put this file in a Quarto project
  (\texttt{\_quarto.yml}), render to \texttt{docs/}, and enable Pages.
\item
\item
  \textbf{Static upload}: Knit to HTML and upload the single HTML file
  to your site.
\item
\end{itemize}

\subsubsection{\texorpdfstring{Minimal \texttt{\_quarto.yml} for a
website}{Minimal \_quarto.yml for a website}}\label{minimal-_quarto.yml-for-a-website}

\begin{verbatim}
project:   type: website   output-dir: docs  website:   title: "Ecological Forecasting Notes"   navbar:     left:       - href: index.qmd         text: Home       - href: gbif-occurrence-workflow.qmd         text: GBIF Workflow  format:   html:     theme: cosmo     toc: true 
\end{verbatim}

\begin{quote}
Save this file next to your \texttt{.qmd}, render with
\texttt{quarto\ render}, then push to GitHub and enable \textbf{GitHub
Pages} (branch \texttt{main}, folder \texttt{/docs}).
\end{quote}
\end{quote}

\end{tcolorbox}




\end{document}
